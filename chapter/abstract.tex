\songti
\begin{center}
  \zihao{-2}\textbf{可视化的Java多线程程序错误定位工具}
\end{center}
\vspace{10mm}
\zihao{4}\textbf{摘要}\par
随着多核CPU的出现,多线程程序在软件开发领域得到广泛的应用。与此同时,多线程程序当中的并发错误也给软件开发者带来的很多困难。2009年,Sangmin Park等人提出了名为Falcon的新方法用来进行多线程程序中并发错误的错误定位。该方法可以按照可疑度的大小报告程序中的数据访问模式。但是,Falcon纯文本的展示方式仍然不便于程序员快速定位错误。所以在本篇论文中,作者首先介绍Falcon的实现原理,然后以Falcon为基础,开发一个可视化的Java多线程程序错误定位工具。
\\[10mm]
\textbf{关键词:错误定位;软件调试;Java;多线程}
\newpage

%English Abstract
\begin{center}
  \zihao{-2}\textbf{Visual Fault Localization Tool for Java Multi-threaded Program}
\end{center}
\vspace{10mm}
\zihao{4}\textbf{Abstract}\par
With the widespread deployment of multi-core processors, multi-thread programming becomes popular in software industry. Meanwhile, concurrent faults in multi-threaded programs cause troubles to software developers. In 2009, Sangmin Park \textit{et al.} presented a new dynamic fault-localization technique to locate concurrent faults in multi-thread programs. It can report data access patterns with suspiciousness scores. However, the results are presented in plain text format which are still hard to read and analyze for programmers. This paper introduces Falcon's algorithm and then implements Falcon approach with a visualization tool.
\\[10mm]
\textbf{Keywords:Fault localization; Software debuging; Java; Multi-threaded}

